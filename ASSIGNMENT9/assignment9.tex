\documentclass[journal,12pt,twocolumn]{IEEEtran}

\usepackage{setspace}
\usepackage{gensymb}

\singlespacing


\usepackage[cmex10]{amsmath}

\usepackage{amsthm}

\usepackage{mathrsfs}
\usepackage{txfonts}
\usepackage{stfloats}
\usepackage{bm}
\usepackage{cite}
\usepackage{cases}
\usepackage{subfig}

\usepackage{longtable}
\usepackage{multirow}

\usepackage{enumitem}
\usepackage{mathtools}
\usepackage{steinmetz}
\usepackage{tikz}
\usepackage{circuitikz}
\usepackage{verbatim}
\usepackage{tfrupee}
\usepackage[breaklinks=true]{hyperref}
\usepackage{graphicx}
\usepackage{tkz-euclide}
\usepackage{float}

\usetikzlibrary{calc,math}
\usepackage{listings}
    \usepackage{color}                                            %%
    \usepackage{array}                                            %%
    \usepackage{longtable}                                        %%
    \usepackage{calc}                                             %%
    \usepackage{multirow}                                         %%
    \usepackage{hhline}                                           %%
    \usepackage{ifthen}                                           %%
    \usepackage{lscape}     
\usepackage{multicol}
\usepackage{chngcntr}

\DeclareMathOperator*{\Res}{Res}

\renewcommand\thesection{\arabic{section}}
\renewcommand\thesubsection{\thesection.\arabic{subsection}}
\renewcommand\thesubsubsection{\thesubsection.\arabic{subsubsection}}

\renewcommand\thesectiondis{\arabic{section}}
\renewcommand\thesubsectiondis{\thesectiondis.\arabic{subsection}}
\renewcommand\thesubsubsectiondis{\thesubsectiondis.\arabic{subsubsection}}


\hyphenation{op-tical net-works semi-conduc-tor}
\def\inputGnumericTable{}                                 %%

\lstset{
%language=C,
frame=single, 
breaklines=true,
columns=fullflexible
}
\begin{document}
\newtheorem{theorem}{Theorem}[section]
\newtheorem{problem}{Problem}
\newtheorem{proposition}{Proposition}[section]
\newtheorem{lemma}{Lemma}[section]
\newtheorem{corollary}[theorem]{Corollary}
\newtheorem{example}{Example}[section]
\newtheorem{definition}[problem]{Definition}

\newcommand{\BEQA}{\begin{eqnarray}}
\newcommand{\EEQA}{\end{eqnarray}}
\newcommand{\define}{\stackrel{\triangle}{=}}
\bibliographystyle{IEEEtran}
\providecommand{\mbf}{\mathbf}
\providecommand{\pr}[1]{\ensuremath{\Pr\left(#1\right)}}
\providecommand{\qfunc}[1]{\ensuremath{Q\left(#1\right)}}
\providecommand{\sbrak}[1]{\ensuremath{{}\left[#1\right]}}
\providecommand{\lsbrak}[1]{\ensuremath{{}\left[#1\right.}}
\providecommand{\rsbrak}[1]{\ensuremath{{}\left.#1\right]}}
\providecommand{\brak}[1]{\ensuremath{\left(#1\right)}}
\providecommand{\lbrak}[1]{\ensuremath{\left(#1\right.}}
\providecommand{\rbrak}[1]{\ensuremath{\left.#1\right)}}
\providecommand{\cbrak}[1]{\ensuremath{\left\{#1\right\}}}
\providecommand{\lcbrak}[1]{\ensuremath{\left\{#1\right.}}
\providecommand{\rcbrak}[1]{\ensuremath{\left.#1\right\}}}
\theoremstyle{remark}
\newtheorem{rem}{Remark}
\newcommand{\sgn}{\mathop{\mathrm{sgn}}}
\providecommand{\abs}[1]{\left\vert#1\right\vert}
\providecommand{\res}[1]{\Res\displaylimits_{#1}} 
\providecommand{\norm}[1]{\left\lVert#1\right\rVert}
%\providecommand{\norm}[1]{\lVert#1\rVert}
\providecommand{\mtx}[1]{\mathbf{#1}}
\providecommand{\mean}[1]{E\left[ #1 \right]}
\providecommand{\fourier}{\overset{\mathcal{F}}{ \rightleftharpoons}}
%\providecommand{\hilbert}{\overset{\mathcal{H}}{ \rightleftharpoons}}
\providecommand{\system}{\overset{\mathcal{H}}{ \longleftrightarrow}}
	%\newcommand{\solution}[2]{\textbf{Solution:}{#1}}
\newcommand{\solution}{\noindent \textbf{Solution: }}
\newcommand{\cosec}{\,\text{cosec}\,}
\providecommand{\dec}[2]{\ensuremath{\overset{#1}{\underset{#2}{\gtrless}}}}
\newcommand{\myvec}[1]{\ensuremath{\begin{pmatrix}#1\end{pmatrix}}}
\newcommand{\mydet}[1]{\ensuremath{\begin{vmatrix}#1\end{vmatrix}}}
\numberwithin{equation}{subsection}
\makeatletter
\@addtoreset{figure}{problem}
\makeatother
\let\StandardTheFigure\thefigure
\let\vec\mathbf
\renewcommand{\thefigure}{\theproblem}
\def\putbox#1#2#3{\makebox[0in][l]{\makebox[#1][l]{}\raisebox{\baselineskip}[0in][0in]{\raisebox{#2}[0in][0in]{#3}}}}
     \def\rightbox#1{\makebox[0in][r]{#1}}
     \def\centbox#1{\makebox[0in]{#1}}
     \def\topbox#1{\raisebox{-\baselineskip}[0in][0in]{#1}}
     \def\midbox#1{\raisebox{-0.5\baselineskip}[0in][0in]{#1}}
\vspace{3cm}
\title{ASSIGNMENT 9}
\author{C.RAMYA TULASI}
\maketitle
\newpage
\bigskip
\renewcommand{\thefigure}{\theenumi}
\renewcommand{\thetable}{\theenumi}
Download all python codes from 
\begin{lstlisting}
https://github.com/CRAMYATULASI/ASSIGNMENT9/tree/main/ASSIGNMENT9/CODES
\end{lstlisting}
%
Latex-tikz codes from 
%
\begin{lstlisting}
https://github.com/CRAMYATULASI/ASSIGNMENT9/tree/main/ASSIGNMENT9
\end{lstlisting}
%
\section{Question No 2.23}
Consider the collision depicted in below fig 1.1 to be between  two billiard balls with equal masses
$m_1=m_2$.The first ball is called the cue while second ball is called the target.The billiards player wants to sink the target ball in corner pocket,which is at an angle $\theta_2=37\degree$.Assume that the collision is elastic and frictional,rotational motions are not important.Obtain $\theta_1$.
\numberwithin{figure}{section}
\begin{figure}[!ht]
    \centering
    \includegraphics[width=\columnwidth]{6.10.png}
    \caption{Collision of two billiard balls}
    \label{fig:.1.1.}
\end{figure}  
\section{SOLUTION}
Let $\vec{v}_{1i}$ and $\vec{v}_{2i}$ be the initial velocities of the first ball and the second ball respectively.
\begin{align}
 \vec{{v}_{1i}} &= \myvec{v_{0}\\0} \\
 \vec{{v}_{2i}} &= \myvec{0\\0} \\
 \vec{{v}_{1f}} &= \myvec{v_{1x}\\v_{1y}}\\
 \vec{{v}_{2f}} &= \myvec{v_{2x}\\2y} 
\end{align}
Because, the target ball initially at rest conversation of energy gives,
\begin{align}
\frac{1}{2}m_1\norm{\vec{v}_{1i}}^2+\frac{1}{2}m_2\norm{\vec{v}_{2i}}^2=\frac{1}{2}m_1\norm{\vec{v}_{1f}}^2+\frac{1}{2}m_2\norm{\vec{v}_{2f}}^2\\
\norm{\vec{v}_{1i}}^2=\norm{\vec{v}_{1f}}^2+\norm{\vec{v}_{2f}}^2
(\because m_1=m_2)\label{2.0.6}
\end{align}
Applying conversation of momentum to two dimensional collision gives,
\begin{align}
\vec{{v}_{1i}}&=\vec{v}_{1f}+\vec{v}_{2f}\quad(\because m_1=m_2)\\
\implies\vec{{v}_{1i}}.\vec{{v}_{1i}}&=(\vec{v}_{1f}+\vec{v}_{2f}).(\vec{v}_{1f}+\vec{v}_{2f})\\
\norm{\vec{v}_{1i}}^2&=\norm{\vec{v}_{1f}}^2+\norm{\vec{v}_{2f}}^2+2(\vec{v_{1f}}.\vec{v_{2f}})\label{2.0.9}
\end{align}
Subtracting \eqref{2.0.6} from \eqref{2.0.9}
\begin{align}
2(\vec{v_{1f}}.\vec{v_{1f}})&=0\\
\implies\norm{\vec{v}_{1f}}\norm{\vec{v}_{2f}}\cos{(\theta_1+\theta_2)}&=0\\
\implies \theta_1+\theta_2 &=90\degree\\
\implies \theta_1+37\degree &=90\degree \quad(\because \theta_2= 37\degree)\\
\implies \theta_1&=53\degree
\end{align}
$\therefore$ Above result shows that whenever two equal masses undergo a glancing elastic collision and one of them is initially at rest they move at right angles to each other.
\end{document}
